\documentclass[12pt]{article}
\usepackage[margin=1in]{geometry}
\usepackage{setspace}
\usepackage{amsmath}
\usepackage{booktabs}
\usepackage{longtable}
\usepackage{graphicx}
\usepackage{natbib}
\usepackage{hyperref}

\onehalfspacing

\title{\textbf{Beyond Objective Measures: \\ 
Subjective Wellbeing and Life Satisfaction of Youth in Chandigarh}}

\author{Nishtha Rajput \\
Master of Development Studies \\
Rajiv Gandhi National Institute of Youth Development}

\date{}

\begin{document}

\maketitle

\begin{abstract}
The study aims to address the gap in understanding of wellbeing from the lens of subjective wellbeing (SWB). Despite India having world’s largest youth population and serving the demographic dividend window, policy inclusivity in respect to wellbeing remains limited with traditional available data using objective and outcome-based approaches. Through stratifies random sampling of 60 youth participants (N=60) aged 18-29 in Chandigarh. This study aims to assess objective wellbeing (OBW), subjective wellbeing (SWB), and identify the key inhibitors and facilitators of subjective wellbeing. This study was conducted through semi-structured interview using two standardised tools Youth Development Index (YDI) for objective wellbeing and Personal Wellbeing Index-Adult (PWI-A) to assess the subjective wellbeing among youth in Chandigarh. Followed by the qualitative questions to identify the inhibitors and facilitators of subjective wellbeing based on PWI triage. Key findings of the study reveal subjective domains of relationship (M=8.83) and community (M=8.86) outweigh other domains of SWB. Highlighting the critical role of social capital in youth well-being. In contrast, the lowest scores were recorded in the Future (M=6.52) and Achieving (M=6.55) domains, signifying youth vulnerabilities. Inhibitors such as Economic constraints, credential devaluation, lack of clarity and highly competitive environment were reported by participants with domain score 0-4. Presence of strong family bond, financial and emotional support, educational attainment and financial freedom are core facilitators among participants with a domain score of 7-10. The study concludes that social relationships and community belongingness are the essence of subjective wellbeing. The observed disparity between high subjective wellbeing M PWI = 7.75 and low objective wellbeing M YDI = 0.51 underscores the necessity of assessing wellbeing through integrated approach for a comprehensive, multidimensional understanding of youth well-being and informing policy that extends beyond traditional indicators such as income and health.

\text{}

\textbf{Keywords:} Subjective wellbeing, Objective wellbeing, Youth wellbeing, Life satisfaction, Personal Wellbeing Index (PWI)

\end{abstract}

\section*{Methodology}

Participants were stratified random samples (N = 60) from age group of 18-29 (M age= 24.06 and SD = 3.07). The youth age range of commonwealth is used after adjusting to the age range of PWI-A. The sample was stratified in two sections. First, based on three markets according to price, i.e. low-price market, middle-price market, and high-price market (1/3 from each) to ensure homogeneity across economic sections. The sample is further equally stratified across two dominant genders, male and female to obtain fair representation of both genders. Responses were collected through face-to-face interview using semi-structured interview schedules based on two standardised tools YDI (Youth development index) and PWI - A (Personal wellbeing Index - Adult). 

\subsection*{Participants}

Participants were stratified random samples (N = 60) from age group of 18-29 (M age= 24.06 and SD = 3.07). The youth age range of commonwealth is used after adjusting to the age range of PWI-A. The sample was stratified in two sections. First, based on three markets according to price, i.e. low-price market, middle-price market, and high-price market (1/3 from each) to ensure homogeneity across economic sections. The sample is further equally stratified across two dominant genders, male and female to obtain fair representation of both genders. Responses were collected through face-to-face interview using semi-structured interview schedules based on two standardised tools YDI (Youth development index) and PWI - A (Personal wellbeing Index - Adult). 

\subsection*{Measures}

The interview schedule used to collect both quantitative and qualitative responses. The interview schedule was divided into three sections, as explained next.

\textit{Section A} includes study background and informed consent.

\textit{Section B} consists of demographic details of participants, such as age, gender, family type, sector of residence, financial source, etc. 

\textit{Section C} registered response on YDI (Youth development index) by Rajiv Gandhi National Institute of Youth Development, 2017 to obtain objective wellbeing to achieve a wholistic view on subjective wellbeing. 

\text{}
Index is adjusted to measure individual values based on self-reporting measures by narrowing the scope from nation aggregates to individual responses. This section has 6 domains (Education, Health and Wellbeing, Employment, Political Participation, Civic Participation, and Social Justice) which are further divided into 18 indicators. To enrich the understanding of social justice, participants were asked if they felt treated unfairly based on caste, religion, gender, or socio-economic background in the past 12 months. If yes, then how?
Section D includes PWI-A by International Wellbeing Group, 2024 to assess subjective wellbeing. PWI-A measure life satisfaction across 7 domains - Standard, Health, Achieving, Relationship, Safety, Community and Future. Additional Global Life Satisfaction (GLS) is included in this section for better understanding. Life satisfaction across domains and GLS is measured using an 11-point Likert scale ranging from 0 (not satisfied at all) to 10 (Completely satisfied). 

Each Likert scale followed by the question “What are the hindrances?” (For PWI 0-4) and 
“What are the driving factors?” (For PWI 7-10) to gain a retrospective view of the inhibitor and facilitators of subjective wellbeing. This enables us to observe domain-wise inhibitors and facilitators for wellbeing.

\subsection*{Data Analysis}

All responses collected through the interview were carefully converted into excel sheets to ensure confidentiality. Quantitative data was analysed using STATA version 15. Descriptive statistical methods were employed to analyse the responses. The PWI score was calculated as per the guidelines by the International Wellbeing Group. Later triage was allocated to PWI and response to interpret as mentioned below.
\begin{itemize}
    \item Score between the 7-10 is Well (W) indicates normal homeostatic control
    \item Score between 5-6 as Under Well (UW) indicates the challenged homeostatic control.
    \item Score below 5 points as No Well (NW) indicates homeostatic failure and psychological distress among the participants.
\end{itemize}
Whereas qualitative data was analyses manually facilitated by Microsoft Excel. Codes were given to the responses, later categorised into themes. The focus was to identify the inhibitors and facilitators as reflected by the participants. 

\section*{Finding and Discussion}
Findings of the study are divided into parts. First, results of objective wellbeing are measured through YDI, along with experiences in social inclusion. Second, results and findings of the subjective wellbeing along with the major inhibitors and facilitators. Third, challenges, facilitators, and suggestions from subjective wellbeing. 


\subsection*{Objective Wellbeing}

Objective wellbeing through YDI was reported as $M_{YDI} = 0.51$ and $SD_{YDI} = 0.10$. The highest performing domain in this is Education M = 0.64 and SD = 0.27 whereas Lowest performance is reported in civic participation M = 0.10 and SD = 0.26. The pull observed in this domain is due to lack of participation in youth programs and NGO activities. 

\text{}

\begin{table}[htbp]
\centering
\begin{tabular}{lcc}
\toprule
Domain & Mean & SD \\
\midrule
Education Index & 0.64 & 0.27 \\
Health and Wellbeing Index & 0.65 & 0.18 \\
Employment Index & 0.45 & 0.23 \\
Political Participation Index & 0.48 & 0.31 \\
Civic Participation Index & 0.10 & 0.26 \\
Social Justice Index & 0.41 & 0.18 \\
Youth Development Index & 0.51 & 0.10 \\
\bottomrule
\end{tabular}
\caption{YDI Scores Across Domains}
\end{table}

Political participation reported as second lowest domain with M = 0.45 and SD = 0.31 pulling indicator in this domain is lack of participation in political activity specially of female participants. Social Justice index scored M = 0.41 and SD = 0.18 shows exclusion based on caste, gender, religion and socio-economic background. The response to the experience of social justice shows the intersectionality of caste-based exclusion, ethno-racial and gender discrimination, appearance-based exclusion, and identity exclusion. “Kudiyan de naal hamesha hi bahut saari cheezan'ch discrimination hundi hai. Te saadi jo jaati hai, oh ST (Scheduled Tribe) hai. Toh as a Scheduled Tribe, aksar ehi sochya janda hai ki, eh bahut backward honge” (Girls are always discriminated against in many things. And our caste is ST. Our caste is ST. So, as a Scheduled Tribe, it is often thought that they would be very backward) (Schedule Caste, Female). A Nepali participant highlighted the intersection of regional and racial identity as a site of exclusion. They are frequently subjected to dehumanizing inquiries regarding their dietary habits and origins: “Log puchte hain... Do you eat dogs?” (People ask us... Do you eat dogs?). This 'othering' is further solidified using racial slurs such as 'Chinki' or 'Momo,' which reduce their complex identity to derogatory labels. One participant recounted an interaction where a peer questioned the competency of officers from Scheduled Caste (SC) backgrounds, attributing their positions solely to reservation. The participant challenged this stereotype, stating: “That's not exactly true. In my opinion, when those people get a job, something good happens. They work harder. Because they've come from poverty. Even if they aren't from poverty, they have faced discrimination (casteism). It happens to almost everyone. So, those people perform well” The interaction ended abruptly with the peer leaving, highlighting the fragility of social cohesion when caste-based meritocratic myths are contested (Schedule caste, Male).

\subsection*{Subjective Wellbeing}
Subjective wellbeing through PWI-A was reported as $M_{PWI}= 7.75$ and $SD_{PWI} = 1.21$. PWI score is calculated as per the guidelines of International Wellbeing Group using the formula given below:
\[
\text{PWI} = \frac{1}{7} \sum_{i=1}^{7} D_i
\]
The PWI score of each domain is depicted in table 2 along with the Global Life Satisfaction (GLS) or “life as a whole”. There is minimal variance in GLS and PWI scores that show reliability and consistency in the response. 

\text{}

\begin{table}[h]
    \centering
    \begin{tabular}{lccc}
    \toprule
       Domain  & Mean & SD \\
    \midrule
       Standard of Living  & 7.58 & 1.80 \\
      Health   & 7.26 & 2.00 \\
      Achieving & 6.55 & 2.73\\
      Relationship & 8.83 & 1.65\\
      Safety & 8.38 & 1.90\\
      Community & 8.86 & 2.14\\
      Future  & 6.52 & 2.31\\
      GLS & 7.88 & 1.86\\
      PWI & 7.75 & 1.21\\
\bottomrule
    \end{tabular}
    \caption{showing the PWI-A score}
    \label{tab:placeholder}
\end{table}

\textbf{Standard of living:} Majority of the participants reported well (7-10). Participants who have scored more due to economic independence, have material security, and basic need fulfillment.  In contrast, participants scored low identify dependency on parents for finances and daily stressors as inhibitors. \textit{“I wake up in the morning, I have everything at home, and my parents are supportive. Dad gave me a car, so I live quite comfortably. As far as studies go, iPhone, everything is there. I'm satisfied.”} (Male, Score - 7).  \textit{“The kind of lifestyle our parents have provided me. Like, I have economic facilities and social facilities. Everything. well-supported financial and a supportive background”} (Female, Score – 8). There is an increase in subjective wellbeing and life satisfaction with a rise in income level due to improved material possession (M. Boo et al., 2020; Diener et al., 2012). However, according to Samuel et al. (2012) there is a mixed effect of wealth including family wealth such as number of cars and computers on wellbeing among adolescents.\textit{“Khud kamati hoon, apne saare kharche khud dekhti hoon, aur kisi pe depend nahi hoon”} (I am happy because I earn myself, I bear all my own expenses, and I am not dependent on anyone, which is why I am happy) (Female, Score - 10). According to Diener et al. (2012), there is a steep and linear pattern observed between the income and SWB. The positive effect of rise in income not only rose the SWB for short duration though sustained for a considerably longer period. 

\text{}

\textbf{Health:} Respondent score low (0-4) attribute time poverty, poor food habits and living conditions to their low wellbeing. According to Lamu and Olsen (2016), health and income significantly impact subjective wellbeing of lower income strata. "I have no control over my eating. It happens, like, you can say, whatever I like, I eat. I don't think about what side effects it will have on my health or anything. If it's sweet, I eat it. And if it's fried too. I mean, I exercise only so that I can eat” (Male, Score - 4). “Kyunki mein ghar se door rehta hooon, nutritious food nhi mil pata” (I live away from home, there is lack of nutritious food) (female, scored 2). “I used to have some or other health related issues very often, like fever is common. I had some liver issue 3 years back because of which I need to be cautious about my intakes” (Female, score –3). 
Whereas absence of disease/illness and health consciousness are facilitators for respondents score high (7-10). Absence of illness/ disease has a positive association with Subjective Wellbeing and increases life satisfaction (Ngamaba et al.,2017); (Adjaye-Gbewonyo et al., 2025); (Wikman et al., 2011) According to Boo et al. (2020) Better health increases satisfaction by 12.80\% and reported as the strongest predictor to life satisfaction and core determinant to wellbeing.  “Health-wise, I am not in prescriptive medicines. I've had regular checkups. I have no allergies... so I have good health” (Male, score - 9). “I used to eat healthy and rich food which helps me.” (Female, score - 10). “I don't fall ill very frequently, also I am not having any deficiencies and live an active life” (Male, score - 9). 

\text{}

\textbf{Achieving:} In contrast to other domains, this domain performed low with M = 6.55. Respondents score low (0-4) consider inhibitors such as Degree without employment value, Economic constraints, Lack of clarity and Peer pressure that affect their wellbeing.\textit{“There was a lack of economic support which affect the achievement.”} (Male, score – 0). \textit{“I don't know what I should be doing, or which thing is right for me. I know what I don't want to do, but I don't know what I want to do.”} (female, score - 0). Wellbeing in this domain is affected positively with the help of family and social support, relational fulfillment, and educational and economic achievement. According to Bücker et al. (2018) there is a small to medium magnitude correlation between academic achievement and wellbeing.“I mean, whatever I have done so far, I have done it with my own hard work. Getting into PGI (Postgraduate Institute of Medical Education and Research) is something I consider achievements” (Female, Score - 9). \textit{“I am working as a government bank employee, earning handsome money, what else I want?”} (Male, Score - 9). Education and occupational achievement are likely to affect wellbeing among adolescents. (Samuel et al., 2012) and being unemployed strongly effect the SWB (Lamu \& Olsen, 2016).

\text{}

\textbf{Relationship:} This domain has the highest among all and has major influence on overall PWI score with M = 8.83 with least SD of 1.65. According to Lamu and Olsen (2016) Social relationship outweigh the effect of income and health in respect to the subjective wellbeing. This shows the importance of relationships and belonginess in youth wellbeing. Family support, Belonginess and attachment, and supportive friend enhance the relationship wellbeing. Quality friendships (Alsarrani et al., 2022) and Positive relationships enhance the level of subjective wellbeing (Moore et al., 2018) even week social interactions help to increase the subjective sense of wellbeing (Sandstrom \& Dunn, 2014). However, Negative experiences such as discrimination and unfair treatment limit subjective wellbeing (Moore et al., 2018). \textit{“Like mainly my parents are much more supportive than I think any other parents are in these current scenarios. So yeah, my parents are way more agreeable to me and my situations”} (Female, score - 10). \textit{“Since we live with friends, we have meals together, we stay together, and I know all their information... so there is a sort of attachment.” }(Male, score - 10).\textit{“I have good friends, and my family is also very supportive. Though I have a single parent, my mother supports me very well in whatever I want to do. I can ask her anything without any hesitation”} (Female, score - 9).

\end{document}
